\chapter{State of the Art}%
\label{chapter:state-of-the-art}



\section{Hydraulic Simulation}

Atualmente no mercado existem diversos simuladores hidráulicos que são utilizados para modelar sistemas de abastecimento de água. Dentre eles, podemos citar o EPANET, WaterGEMS, WaterCAD, entre outros. A simulação permite oferecer soluções rápidas e precisas para equações algébricas diferenciais usadas para desenvolver a representação matemática de um sistema de abastecimento de água \cite{rfc1}. E o EPANET é um dos mais reconhecidos aplicativos na área de distribuição de água \cite{rfc4}, e é esta ferramenta a ser utilizada na simulação.

A calibração de um modelo hidráulico é uma tarefa desafiadora pois necessita de um grande número de parâmetros incertos\cite{rfc3}. Isto é necessário para possibilitar que os resultados obtidos sejam confiáveis. 

As simulações hidráulicas demandam muitos mais dados para o ajuste no processo de calibração dos componentes. 


\section{Deep Leaning in WSS}

O método de operação tradicional baseado na experiência humana precisa que operadores para regular os equipamentos presentes nos sistemas de abastecimento de água. Geralmente leva-se muito tempo para sua conclusão.  (Artificial neural network-based water distribution scheme in real-time in long-distance water supply systems).

Além disso, um processo de calibração é necessário antes de usar modelos hidráulicos, o que envolve ajustar repetidamente muitos parãmetros para minimizar o erro entre os valores simulados e observados (Meirelles et al. 2017; Lima et al. 2018).

Com o progresso contínuo da tecnologia de inteligência artificial e métodos de aprendizado de máquina (ML), 
recorre-se cada vez mais a metamodelos, que também são conhecidos como modelos substitutos (Razavi et al. 2012). Estes modelos podem substituir modelos computacionais custosos uma vez que as correlaçoes entre entradas e saídas são estabelecidas com base em dados disponíveis (Broad et al. 2010). O principal objetivo dos metamodelos é reduzir o esforço computacional necessário pelos modelos hidráulicos (Pasha and Lansey, 2014; Dini and Tabesh, 2019)

Inicialmente, a regressão linear era usada para estimar essas correlações, mas abordagens modernas agora utilizam redes neurais artificiais (ANNs) e teorias de ML, devido à sua forte capacidade de expressão não-linear (Romano and Kapelan, 2014; LeCun et al., 2015). O modelo de ANN mais amplamente usado atualmente é o multi-layer perceptron (MLP), que consiste em camadas totalmente conectadas (Hu et al., 2019; Garzon et al., 2022).





\section{Optimization in WSS}

O principal objetivo para otimização é o custo operacional, que compreende o custo da energia elétrica e o custo de manutenção das bombas \cite{rfc2} (Pump Scheduling Optimization Model for Water Supply System Using AWGA). Os sistemas de bombeamento consomem a maior quantidade de energia em sistemas de abastecimento de água, geralmente respondendo por mais de 80\% do consumo total de energia, de acordo com Sarbu (2016).


A otimização do custo de energia em sistemas de abastecimento de água pode ser alcançada por meio de várias medidas, como substituição de bombas, mudança na operação da estação de bombeamento, modernização do sistema de tubulações, modelagem computacional de mudanças na operação da rede e seleção de soluções que garantam os melhores efeitos econômicos e técnicos (Świe¸tochowska and Bartkowska, 2022).


teste~\cite{rfc1}

A otimização dos WSS apresenta inúmeras aplicações, mas é crucial enfatizar a importância da otimização do agendamento de bombas devido ao considerável consumo de energia associado a este componente essencial do WSS (Cost efficiency in water supply systems: An applied review on optimization models for the pump scheduling problem)

(...continuar)

\section{Hard Constraints}

\section{Optimization Process}

\subsection{Mathematical Formulation}

\subsection{Optimization Methods}

