\chapter{State of the Art}%
\label{chapter:state-of-the-art}



\section{Hydraulic Simulation}

\begin{comment}
Atualmente no mercado existem diversos simuladores hidráulicos que são utilizados para modelar sistemas de abastecimento de água. Dentre eles, podemos citar o EPANET, WaterGEMS, WaterCAD, entre outros. A simulação permite oferecer soluções rápidas e precisas para equações algébricas diferenciais usadas para desenvolver a representação matemática de um sistema de abastecimento de água\cite{rfc1}. O EPANET é um dos mais reconhecidos aplicativos na área de distribuição de água\cite{rfc4}, amplamente utilizaddo e com muito sucesso\cite{rfc5} há mais de 20 anos\cite{rfc6} e é esta ferramenta a ser utilizada na simulação para obtenção dos dados necessários para a otimização.    
\end{comment}


Currently, there are several hydraulic simulators on the market used to model water supply systems, such as EPANET, WaterGEMS, and WaterCAD. Simulation provides quick and accurate solutions to differential algebraic equations for developing a mathematical representation of a water supply system\cite{rfc1}. EPANET is one of the most recognized applications in water distribution\cite{rfc4}, widely used and quite successful\cite{rfc5} for over 20 years\cite{rfc6}, and this tool will be used to obtain the data needed for optimization.


\section{Deep Leaning}


\subsection{Neural network}

The main concept of neural networks (NN) was proposed and introduced as a mathematical model of an artificial neuron in 1943\cite{rfc23}.

Basics components of a neural network architecture are neurons, activations, biases, weight and layers - in order to define the multilayer perceptron (MLP) model.\cite{rfc23}


## INCLUIR FORMULA ## 
The most basic component of the neural network is the neuron and consists of two simple operations: the preactivation zi of a neuron is a linear aggregation of incoming signals, where each signal is weighted by Wij and biased by bi. Each neuron then fires or not according to the weighted and biased evidence, according to the values of the preactivation zi, and produces an activation.

The scalar-valued function is called the activation function and acts independently on each component of the preactivation vector. 

Taken together, these neurons form a layer.\cite{rfc27}

For the activation function, the most common are the sigmoid, tanh, sin, linear, ReLU, GELU and other functions.

\subsection{Deep Learning}

Deep learning is a name for a specific class of artificial neural networks, which in turn are a special class of machine learning algorithms, applicable to natural language processing, computer vision, robotics and many more.\cite{rfc22}

Traditional Machine Learning algorithms use handwritten feature extraction to train algorithms, while Deep Learning algorithms use modern techniques to extract these features in an automatic fashion.\cite{rfc23}

It has been demonstrated unequivocally that multilayered artificial neural architectures can learn complex, non-linear functional mappings, given sufficient computational resources and training data. Importantly,unlike more traditional approaches, their results scale with training data.\cite{rfc24}

Deep learning provides a black-box method to learn from complex, high-dimensional data in order to infer robust and scalable insights, minimizing the degree of manual work required\cite{rfc23}. A feature that sets deep learning apart from its broader field of machine learning is the use of multilayer models that lead to a higher-level representation of the underlying data sources\cite{rfc24}.

Neural networks are capable of learning—by changing the distribution of weights it is possible to approximate a function representative of the patterns in the input. The key idea is to re-stimulate the black-box using new excitation (data) until a sufficiently well-structured representation is achieved. Each stimulation redistributes the neural weights a little bit—hopefully in the right direction, given the learning algorithm involved is appropriate for use, until the error inapproximation with respect to some well-defined metricis below a practitioner-defined lower bound. Learning then,is the aggregation of avariable length of causal chains of neural computations\cite{rfc26} seeking to approximate a certain pattern recognition task through linear/nonlinear modulation of the activation of the neurons across the architecture.

The use of Deep Leaening has grown tremendously in the last few years with the rise of GPUs, big data, cloud providers such as Amazon Web Services (AWS), Microsoft Azure and Google Cloud, and frameworks such as PyTorch, Keras, TensorFlow and many others. In addition to this, large companies share algorithms trained on huge datasets, thus helping startups to build state-of-the-art systems on several use cases with little effort.\cite{rfc23}


\subsection{Deep Learning in WSS}


\begin{comment}
A crescente demanda por eficiência e sustentabilidade nos sistemas de abastecimento de água tem impulsionado o uso de Machine Learning (ML) e Deep Learning (DL) como ferramentas inovadoras para otimização, monitoramento e controle dessas redes. Com os dados obtidos por ferramentas como EPANET, é possível treinar modelos de aprendizado de máquina para prever o comportamento do sistema e otimizar o uso de recursos.

Há diversos papers abordando o uso de Deep Learning em sistemas de abastecimento e distribuição de água para prever detecção de anomalias\cite{rfc12}, prever pressão na rede de suprimento de água\cite{rfc11}, detecção de vazamentos\cite{rfc13}, agendamento inteligente para ligar e desligar a bomba, de acordo com o monitoramento em tempo real\cite{rfc9} entre outros.
    
\end{comment}





The increasing demand for efficiency and sustainability in water supply systems has driven the use of Machine Learning and Deep Learning as innovative tools for optimization, monitoring, and control of these networks. With data obtained from tools like EPANET, it is possible to train machine learning models to predict system behavior and optimize resource usage.

There are several papers addressing the use of Deep Learning in water supply and distribution systems to predict anomaly detection\cite{rfc12}, predict pressure in the water supply network\cite{rfc11}, detect leaks\cite{rfc13}, and smart scheduling to turn the pump on and off according to real-time monitoring\cite{rfc9}, among others.


\section{Optimization in WSS}

\begin{comment}

De uma forma mais técnica, diz-se que a otimização é o processo de maximizar ou minimizar a função-objetivo requerida enquanto são satisfeitas determinadas restrições.\cite{rfc21}

Estações de bombeamento de abastecimento de água são um elemento indispensável de qualquer sistema de abastecimento de água. Elas fornecem não apenas o suprimento de água para cada recipiente, mas também a pressão necessária na rede de abastecimento de água para fins de combate a incêndio. O custo da energia é um dos componentes mais importantes do preço da água tratada.\cite{rfc7}

O principal objetivo para otimização é o custo operacional, que compreende o custo da energia elétrica e o custo de manutenção das bombas.\cite{rfc2} Os sistemas de bombeamento consomem a maior quantidade de energia em sistemas de abastecimento de água, geralmente respondendo por mais de 80\% do consumo total de energia.\cite{rfc8}

A otimização do custo de energia em sistemas de abastecimento de água pode ser alcançada por meio de várias medidas, como substituição de bombas, mudança na operação da estação de bombeamento, modernização do sistema de tubulações, modelagem computacional de mudanças na operação da rede e seleção de soluções que garantam os melhores efeitos econômicos e técnicos.\cite{rfc7}

A otimização dos WSS apresenta inúmeras aplicações, mas é crucial enfatizar a importância da otimização do agendamento de bombas devido ao considerável consumo de energia associado a este componente essencial do WSS (Cost efficiency in water supply systems: An applied review on optimization models for the pump scheduling problem)

Os sistemas de bombeamento têm um potencial significativo para melhorias de eficiência energética. Em muitos casos, a otimização das operações consideram a velocidade dixa da bombas e a economia de custos pode ser obtida utilizando o padrão de variação do custo da tarifa de energia elétrica ao longo do dia.\cite{rfc8}

As variações horárias na demanda de água durante o dia são muito maiores em comparação à demanda média diária. Para um consumidor doméstico, a necessidade de água é maior durante as horas da manhã e da noite do que a demanda do meio-dia. Durante os horários de pico, o custo de energia é 2 a 3 vezes mais caro do que durante os horários de consumo mínimo.

Uma solução técnica para essa redução pode ser uma diminuição na potência de bombeamento (mesmo parando bombas se for possível) durante os horários de pico, juntamente com uma entrega extensiva fora desses horários. Consequentemente, os sistemas de distribuição devem ser equipados com tanques de armazenamento.


\end{comment}

Optimization is applied in many systems and situations, thus making it an important paradigm in technology. When we try to optimize, we either minimize (resource consumption, cost) or maximize (profit, system performance)\cite{rfc19}.

In a more technical sense, optimization is said to be the process of maximizing or minimizing the required objective function while satisfying certain constraints.\cite{rfc21}

Water supply pumping stations are an indispensable element of any water supply system. They not only provide the water supply to each container but also the necessary pressure in the water supply network for firefighting purposes. The cost of energy is one of the most important components of the price of treated water.\cite{rfc7}

The main objective for optimization is the operational cost, which encompasses the cost of electric power and pump maintenance.\cite{rfc2} Pumping systems consume the largest amount of energy in water supply systems, generally accounting for more than 80\% of the total energy consumption.\cite{rfc8}

Energy cost optimization in water supply systems can be achieved through several measures, such as pump replacement, changes in pumping station operations, pipe system modernization, computational modeling of network operation changes, and the selection of solutions that ensure the best economic and technical effects.\cite{rfc7}

WSS optimization has numerous applications, but it is crucial to highlight the importance of pump scheduling optimization due to the considerable energy consumption associated with this essential WSS component (Cost efficiency in water supply systems: An applied review on optimization models for the pump scheduling problem)

Pumping systems have significant potential for energy efficiency improvements. In many cases, operations optimization considers fixed pump speed, and cost savings can be obtained by using the variation in electricity tariff rates throughout the day.\cite{rfc8}

Hourly variations in water demand during the day are much higher compared to the average daily demand. For a household consumer, water demand is higher during morning and evening hours than midday. During peak hours, energy costs can be 2 to 3 times more expensive than during off-peak hours.

A technical solution for this reduction may involve decreasing pumping power (even stopping pumps if possible) during peak hours, along with extensive delivery outside these hours. Consequently, distribution systems must be equipped with storage tanks.


\section{Hard Constraints}

\begin{comment}
    
Hard constraints é um termo utilizado para verificar se certas propriedades são satisfeitas para resolução de problemas de otimização.\cite{rfc16} São como regras não negociáveis.

A satisfação das restrições corresponde à aplicação dos limites de segurança operacional e à adesão às leis físicas, sendo de suma importância\cite{rfc15}.

Algoritmos de aprendizado baseados em gradiente permitem otimizar os parâmetros de redes para aproximar qualquer modelo desejado. No entanto, apesar da existência de vários algoritmos avançados de otimização, a questão de impor restrições estritas de igualdade durante o treinamento não foi suficientemente abordada\cite{rfc10}.

Aplicar solucionadores tradicionais para otimização geral restrita, como SQP\cite{rfc18}, a redes neurais pode ser não trivial. Como os métodos tradicionais expressam restrições como uma função de parâmetros aprendíveis, essa formulação se torna extremamente dimensional, não linear e não convexa no contexto de redes neurais\cite{rfc10}.

Em operações de sistemas de abastecimento de água é restringida por requisitos mínimos de pressão; limitações de capacidade impostas por bombas, dutos e tanques; e um conjunto de restriçõies hidráulicas. As restrições hidráulicas que dão origem a complexas formulações mistas inteiras e não lineares. A primeira classe de métodos impõe restrições de pressão e capacidade explicitamente, enquanto as restrições hidráulicas são inlcuídas implicitamente por meio de ferramentas de simulação de rede de água, como o EPANET. \cite{rfc17}

\end{comment}

Hard constraints is a term used to check whether certain properties are satisfied for solving optimization problems.\cite{rfc16} They are like non-negotiable rules.

Meeting constraints corresponds to applying operational safety limits and adhering to physical laws, which is of great importance.\cite{rfc15}

Gradient-based learning algorithms allow optimizing network parameters to approximate any desired model. However, despite the existence of several advanced optimization algorithms, the issue of imposing strict equality constraints during training has not been sufficiently addressed.\cite{rfc10}

Applying traditional solvers for general constrained optimization, such as SQP\cite{rfc18}, to neural networks can be non-trivial. Since traditional methods express constraints as a function of learnable parameters, this formulation becomes extremely high-dimensional, non-linear, and non-convex in the context of neural networks.\cite{rfc10}

Water supply system operations are constrained by minimum pressure requirements; capacity limitations imposed by pumps, pipes, and tanks; and a set of hydraulic constraints. These hydraulic constraints give rise to complex mixed-integer and non-linear formulations. The first class of methods explicitly imposes pressure and capacity constraints, while hydraulic constraints are implicitly included through water network simulation tools, such as EPANET.\cite{rfc17}

\section{Optimization Process}

\subsection{Mathematical Formulation}

\subsection{Optimization Methods}

