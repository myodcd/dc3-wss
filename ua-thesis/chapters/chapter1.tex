\chapter{Introduction}%


\section{Background}

\begin{comment}

Um recurso finito e essencial para existência humana é a água. Por tratar-se de um item de extrema importância, é necessário que haja um controle e monitoramento eficaz de seu uso. Mas não só do seu uso mas da energia e custos relacionados ao seu fornecimento à população.

Hoje temos acesso a diversas tecnologias que podem ser utilizadas para otimizar o uso da água e reduzir custos, como por exemplo, algoritmos de Machine Learning para prever o consumo de água, a utilização de sensores para monitorar o consumo, a calibração dos modelos hidráulicos, entre outros.

Nos WSS (Water Supply Systems), a otimização de custos é um desafio, pois é necessário considerar diversos fatores, como a topologia da rede, a demanda de água, a localização dos reservatórios. Outra dificuldade está na consideração da incerteza dos dados devido a muitos casos, os dados coletados são imprecisos ou incompletos.

No campo computacional, a complexidade de cálculos ou a falta de dados gerados na fonte não são grandes desafios como eram a alguns anos atrás. Com a IoT (Internet of Things) e a computação em nuvem, é possível coletar e processar uma grande quantidade de dados em tempo real. Esta quantidade de dados (Big Data) pode ser utilizada para treinar algoritmos de Machine Learning e otimizar custos em WSS. 

Diante destes desafios, a tecnologia é uma aliada nestes vários aspectos a serem tratados. 


\end{comment}

A finite and essential resource for human existence is water. Because it is of utmost importance, effective control and monitoring of its use are required. Moreover, it is necessary to consider not only its usage but also the energy and costs related to supplying it to the population.

Nowadays, various technologies can be employed to optimize water usage and reduce costs, such as Machine Learning algorithms to predict consumption, the use of sensors to monitor it, hydraulic model calibration, among others.

In WSS (Water Supply Systems), cost optimization is challenging because several factors must be considered, including network topology, water demand, and reservoir location. Another difficulty stems from data uncertainty, as in many cases, the collected data are inaccurate or incomplete.

In the computational field, the complexity of calculations or the lack of source-generated data are no longer as significant hurdles as they were a few years ago. With IoT (Internet of Things) and cloud computing, it is possible to collect and process vast amounts of data in real time. This dataset (Big Data) can be used to train Machine Learning algorithms and optimize costs in WSS.

Given these challenges, technology is an ally in addressing the various aspects involved.

\section{Aim}

\begin{comment}
Na literatura, encontram-se implementações no uso de diversos algoritmos de Machine Learning para simular e otimizar custos de sistemas de abastecimento de água.

A proposta do presente artigos é utilizar o algoritmo DC3 (Deep Constraint Completion and Correction), como forma de apresentar um novo método para simulação e otimização e comparar os resultados obtidos com os algoritmos já existentes, utilizando como fontes de dados os sistemas de abastecimento de água de Fontinha e Ronqueira.
    
\end{comment}


In the literature, there are implementations that use various Machine Learning algorithms to simulate and optimize costs in water supply systems.

The goal of this article is to employ the DC3 (Deep Constraint Completion and Correction) algorithm as a new method for simulation and optimization and to compare the results obtained with existing algorithms, using the Fontinha and Ronqueira water supply systems as data sources.

\section{Reading guide}
This work is structured into 5 chapters:

\begin{itemize}

\item Chapter 1: Introduction to the theme, addressing its importance, the problems to be solved, as well as a possible solution, the aim of the work, and the structure of the dissertation.

\item Chapter 2: Literature review. It includes hydraulic simulation using the EPANET tool, the use of Deep Learning in water supply systems, optimization, hard constraints, and the optimization process.

\item Chapter 3: Methodology used for cost optimization in water supply systems with the DC3 (Deep constraints correction and completion) algorithm.

\item Chapter 4: Implementation of the techniques and interpretation of the results obtained in each of the observed case studies.

\item Chapter 5: Presentation of conclusions summarizing all the work carried out and suggestions for future studies.

\end{itemize}
