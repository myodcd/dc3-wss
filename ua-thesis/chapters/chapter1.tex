\chapter{Introduction}

\section{Background}

Water is a finite and essential resource for human existence. Due to its vital importance, it requires effective control and monitoring of its use. However, beyond the management of consumption, attention must also be given to the energy demand and operational costs associated with water supply to the population.

In recent years, several technologies have emerged to enhance water management and reduce operational costs. Examples include the application of Machine Learning algorithms to forecast consumption, the use of sensor networks for continuous monitoring, and the calibration of hydraulic models to improve system accuracy.

In Water Supply Systems (WSS), cost optimization remains a complex challenge. It involves multiple interdependent factors, such as network topology, water demand variability, and reservoir configuration. Furthermore, the presence of uncertainty and noise in measurement data—often incomplete or imprecise—makes accurate modeling and optimization even more difficult.

From a computational perspective, many of the limitations once imposed by data scarcity and processing complexity have been mitigated. The advent of the Internet of Things (IoT) and the widespread availability of cloud computing now allow the collection and real-time processing of large volumes of data. This so-called “Big Data” can be exploited to train advanced Machine Learning models capable of supporting decision-making and improving cost efficiency in WSS operations.

In this context, technology becomes a fundamental ally for addressing the various challenges involved in the management and optimization of water distribution systems.

\section{Aim}

Several studies in the literature report the use of different Machine Learning algorithms to simulate and optimize the operation of water supply systems, focusing primarily on minimizing energy consumption and operational costs.

The present dissertation proposes the use of the DC3 (Deep Constraint Completion and Correction) algorithm as a novel approach to cost optimization in water supply systems. The method will be applied to the Fontinha and Ronqueira systems, enabling the comparison of its performance with existing optimization algorithms. The main contribution lies in evaluating the ability of DC3 to handle hard constraints while maintaining feasible and near-optimal solutions.

\section{Reading guide}

This work is organized into five chapters:

\begin{itemize}
    \item \textbf{Chapter 1} introduces the research topic, its relevance, the problem statement, and the overall aim of the dissertation, as well as an outline of its structure.
    
    \item \textbf{Chapter 2} presents a literature review covering hydraulic simulation with the EPANET tool, the application of Deep Learning in water supply systems, optimization approaches, hard constraints, and general optimization processes.
    
    \item \textbf{Chapter 3} describes the methodology developed for cost optimization in water supply systems using the DC3 (Deep Constraint Completion and Correction) algorithm.
    
    \item \textbf{Chapter 4} details the implementation of the proposed techniques and discusses the results obtained from each case study.
    
    \item \textbf{Chapter 5} provides the main conclusions, summarizing the key findings of the research and suggesting directions for future work.
\end{itemize}
