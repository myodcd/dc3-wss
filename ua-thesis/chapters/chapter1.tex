\chapter{Introduction}%

\begin{introduction}
A short description of the chapter.

A memorable quote can also be used.
\end{introduction}



\section{Background}
Um recurso finito e essencial para existência humana é a água. Por tratar-se de um item de extrema importância, é necessário que haja um controle e monitoramento eficaz de seu uso. Mas não só do seu uso mas da energia e custos relacionados ao seu fornecimento à população.

Hoje temos acesso a diversas tecnologias que podem ser utilizadas para otimizar o uso da água e reduzir custos, como por exemplo, algoritmos de Machine Learning para prever o consumo de água, a utilização de sensores para monitorar o consumo, a calibração dos modelos hidráulicos, entre outros.

Nos WSS (Water Supply Systems), a otimização de custos é um desafio, pois é necessário considerar diversos fatores, como a topologia da rede, a demanda de água, a localização dos reservatórios. Outra dificuldade está na consideração da incerteza dos dados devido a muitos casos, os dados coletados são imprecisos ou incompletos.

No campo computacional, a complexidade de cálculos ou a falta de dados gerados na fonte não são grandes desafios como eram a alguns anos atrás. Com a IoT (Internet of Things) e a computação em nuvem, é possível coletar e processar uma grande quantidade de dados em tempo real. Esta quantidade de dados (Big Data) pode ser utilizada para treinar algoritmos de Machine Learning e otimizar custos em WSS. 

Diante destes desafios, a tecnologia é uma aliada nestes vários aspectos a serem tratados. 

\section{Aim}

Na literatura, encontram-se implementações no uso de diversos algoritmos de Machine Learning para simular e otimizar custos de sistemas de abastecimento de água.

A proposta do presente artigos é utilizar o algoritmo DC3 (Deep Constraint Completion and Correction), como forma de apresentar um novo método para simulação e otimização e comparar os resultados obtidos com os algoritmos já existentes, utilizando como fontes de dados os sistemas de abastecimento de água de Fontinha e Ronqueira.